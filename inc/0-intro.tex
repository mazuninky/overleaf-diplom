\likechapter{Введение}

Взаимоблокировка или Deadlock \cite{Deadlock} — это ситуация в многозадачной среде, при которой несколько процессов или потоков находятся в состоянии ожидания ресурсов, занятых друг другом, при этом ни один из них не может продолжать свое выполнение. Данная проблема встречается часто в многопоточных приложения и может не только снижать производительность но и приводить к полному “зависанию” всей системы в целом.

Первые компьютеры были однопоточными и программистам не приходилось заботиться о взаимодействие нескольких потоков между собой. С усложнением приложений и ростом требований к системам, потребовались многопроцессные и многопоточные операционные системы.

\textbf{Актуальность темы исследования.} Актуальность темы дипломной работы связана с повсеместным использованием многопточных систем и заключается в необходимости развития инструментов для тестирования программного кода для многопоточных систем\dots

\textbf{Объект исследования} - многопоточные среды

\textbf{Предмет исследование} - алгоритмы обнаружения взаимоблокировок в мнопоточных средах

\textbf{Гипотеза:} Предполагается, что в ходе данной работы получиться улучшить один из параметров алгоритма, без ухудшения других:
\begin{itemize} 
\item  производительность(обратный параметр, процентное соотношение скорости исходной работы и после инструментации)
\item количество ошибок(уменьшение ошибок первого и второго родов)
\item используемая память(обратный параметр, процентное соотношение используемой памяти исходной работы и после инструментации).
\end{itemize} 

\textbf{Цель} - улучшение алгоритма детектирования Deadlock

В связи с поставленно целью были выявлены следующие \textbf{задачи}:

\begin{itemize}  
\item Обзор алгоритмов и подходов к обнаружению взаимоблокировок в многопоточных средах
\item Разработка алгоритма для обнаружению взаимоблокировок
\item Реализация алгоритма для обнаружению взаимоблокировок
\item Тестирование производительности разработанного алгоритма
\end{itemize}

В процессе работы над дипломом были использованы следующие методы исследования: изучение материалов научных и периодических изданий по проблеме, анализ документации и нагрузочное тестирование.

\textbf{Научная новизна} состоит в использование новых подходов 

\clearpage