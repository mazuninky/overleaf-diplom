%%% Преамбула %%%

\usepackage[T2A]{fontenc}   % кодировка
\usepackage[utf8]{inputenc} % кодировка исходного текста
\usepackage[russian]{babel} % локализация и переносы
\usepackage{extsizes}       % размер шрифта 

\usepackage{subcaption}
\usepackage{multicol}
\usepackage{float}
\usepackage{listings} % листинг кода
\captionsetup{compatibility=false}
\usepackage{algorithm}
\usepackage{algpseudocode}

% далее две строки отвечающие за т.н. "полуторный интервал"
% плюс исправляем кабанические отступы вокруг формул
\usepackage[nodisplayskipstretch]{setspace}
\onehalfspacing
\parskip 1.5ex % paragraph spacing

\usepackage{graphicx} % для вставки картинок
\usepackage{amssymb,amsfonts,amsmath,amsthm, mathtools} % математические дополнения от АМС
\usepackage{icomma} % умная запятая)) $3,49$ - число, $3, 49$ - перечисление
\usepackage{indentfirst} % отделять первую строку раздела абзацным отступом тоже
\usepackage[usenames,dvipsnames]{color} % названия цветов
\usepackage{makecell}
\usepackage{multirow} % улучшенное форматирование таблиц
\usepackage{ulem} % подчеркивания

\usepackage{euscript} % математический шрифт Евклид (красивый) для формул
\usepackage{mathrsfs} % красивые вычурные буквы с завитушками для формул
% Перенос знаков в формулах по Львовскому (дублировать знак при переносе)
\newcommand*{\hm}[1]{#1\nobreak\discretionary{}
            {\hbox{\mathsurround=0pt #1}}{}}
\usepackage{physics} % для удобной записи производных

% графики
\usepackage{tikz}

% ====================================================================
%                   Настраиваем стандартные поля
%                   Верхнее и нижнее - 2см
%                   Левое - 3см
%                   Правое - 1.5см
% ====================================================================
\usepackage{geometry}
\geometry{left=3cm}
\geometry{right=1.5cm}
\geometry{top=2cm}
\geometry{bottom=2cm}

% ====================================================================
%                       Настройка заголовков
% ====================================================================
\usepackage{titlesec}
 
\titleformat{\chapter}[display]
    {\filcenter}
    {\MakeUppercase{\chaptertitlename} \thechapter}
    {8pt}
    {\bfseries}{}
 
\titleformat{\section}
    {\normalsize\bfseries}
    {\thesection}
    {1em}{}
 
\titleformat{\subsection}
    {\normalsize\bfseries}
    {\thesubsection}
    {1em}{}
% Настройка вертикальных и горизонтальных отступов заголовков
\titlespacing*{\chapter}{0pt}{-30pt}{8pt}
\titlespacing*{\section}{\parindent}{*4}{*1.3} % мне отступы снизу показались слишком большими
\titlespacing*{\subsection}{\parindent}{*4}{*1.3}


% ====================================================================
%                     Включаем Times New Roman
% ====================================================================
\renewcommand{\rmdefault}{ftm} 
\frenchspacing

% ====================================================================
%           Нумерация страниц справа снизу (контитулы)
% ====================================================================
\usepackage{fancyhdr}
\pagestyle{fancy}
\fancyhf{}
\fancyfoot[R]{\thepage}
\fancyheadoffset{0mm}
\fancyfootoffset{0mm}
\setlength{\footskip}{17pt}
\renewcommand{\headrulewidth}{0pt}
\renewcommand{\footrulewidth}{0pt}
\fancypagestyle{plain}{ 
    \fancyhf{}
    \rfoot{\thepage}}
\setcounter{page}{3} % начать нумерацию страниц с №3

% ====================================================================
%                       Настройка оглавления
% ====================================================================
\usepackage{tocloft}
\renewcommand{\cfttoctitlefont}{\hspace{0.38\textwidth} \bfseries\MakeUppercase}
\renewcommand{\cftbeforetoctitleskip}{-1em}
\renewcommand{\cftaftertoctitle}{\mbox{}\hfill \\ \mbox{}\vspace{-2.5em}}
\renewcommand{\cftchapfont}{\normalsize\bfseries \MakeUppercase{\chaptername} }
\renewcommand{\cftsecfont}{\hspace{31pt}}
\renewcommand{\cftsubsecfont}{\hspace{0pt}} % убрал отступ подпараграфов в расписании
\renewcommand{\cftbeforechapskip}{1em}
\renewcommand{\cftparskip}{1mm}
\renewcommand{\cftdotsep}{1}
\setcounter{tocdepth}{2} % задать глубину оглавления — до subsection включительно

% ====================================================================
%                   Настройка специальных разделов
% Встроим в оглавление специальные главы: аннотацию, введение и т.д.
% Такие главы будут создаваться командой "\likechapter{название главы}" 
% ====================================================================
\newcommand{\likechapterheading}[1]{ 
    \begin{center}
    \textbf{\MakeUppercase{#1}}
    \end{center}}

\makeatletter
    \renewcommand{\@dotsep}{2}
    \newcommand{\l@likechapter}[2]{{\bfseries\@dottedtocline{0}{0pt}{0pt}{#1}{#2}}}
\makeatother
\newcommand{\likechapter}[1]{    
    \likechapterheading{#1}    
    \addcontentsline{toc}{likechapter}{\MakeUppercase{#1}}}
    
% % ====================================================================
% %                 Организация рабочего пространства
% % ====================================================================
% \usepackage{import}
% \newcommand{\includechapter}[1]{\subimport{chapter_#1/}{chapter_#1}\newpage}
% \newcommand{\includeappend}[1]{\subimport{append_#1/}{append_#1}\newpage}
% \usepackage{tabularx}

% ====================================================================
%                         Список литературы
% ====================================================================
\bibliographystyle{gost-numeric.bbx}
\usepackage[parentracker=true, backend=biber, hyperref=false, bibencoding=utf8, style=numeric-comp, language=auto, autolang=other, citestyle=gost-numeric, defernumbers=true, bibstyle=gost-numeric, sorting=ntvy]{biblatex}
\addbibresource{sources.bib}


% ====================================================================
%                           Приложения
% ====================================================================
\usepackage[title,titletoc]{appendix}
\titleformat{\paragraph}[display]
    {\filcenter}
    {\MakeUppercase{\chaptertitlename} \thechapter}
    {8pt}
    {\bfseries}{}
\titlespacing*{\paragraph}{0pt}{-30pt}{8pt}
\newcommand{\append}[1]{
    \clearpage
    \stepcounter{chapter}
    \begin{center}
        \paragraph{\MakeUppercase{\chaptertitlename~\Asbuk{chapter}}}
        \vspace{-0.5cm}
        \paragraph{\MakeUppercase{ #1}}
    \end{center}
    \addcontentsline{toc}{likechapter}{\MakeUppercase{\chaptertitlename~\Asbuk{chapter}: #1}}
}

% ====================================================================
%                           МЕТА-ДАННЫЕ
% ====================================================================
\usepackage{authblk}
\author{Мазунин К.Ю.}
\title{Алгоритмы обнаружения взаимоблокировок в многопоточных приложениях}
\affil{Университет "ИТМО"}